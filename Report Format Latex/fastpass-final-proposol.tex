%%%%%% Run at command line, run
%%%%%% xelatex grad-sample.tex 
%%%%%% for a few times to generate the output pdf file
\documentclass[12pt,oneside,openright,a4paper]{cpe-thai-project}


\usepackage{polyglossia}
\setdefaultlanguage{thai}
\setotherlanguage{english}
\newfontfamily\thaifont[Script=Thai,Scale=1.23]{TH Sarabun New}
\defaultfontfeatures{Mapping=tex-text,Scale=1.23,LetterSpace=0.0}
\setmainfont[Scale=1.23,LetterSpace=0,WordSpace=1.0,FakeStretch=1.0,Mapping=tex-text]{TH Sarabun New}
\XeTeXlinebreaklocale "th"	
\XeTeXlinebreakskip = 0pt plus 0pt
\emergencystretch=10pt

%%%%%%%%%%%%%%%%%%%%%%%%%%%%%%%%%%%%%%%%%%%%%%%%%%%%%%%%%%%%%%%%%%%
% Customize below to suit your needs 
% The ones that are optional can be left blank. 
%%%%%%%%%%%%%%%%%%%%%%%%%%%%%%%%%%%%%%%%%%%%%%%%%%%%%%%%%%%%%%%%%%%
% First line of title
\def\disstitleone{ระบบศูนย์กลางที่จอดรถอัจฉริยะ(Smart Central Parking Hub)}   
% Second line of title
%\def\disstitletwo{Project/Indep title line 2 (optional)}   
% Your first name and lastname
\def\dissauthor{ นายกันต์ดนัย ศรีวัฒนะ\qquad65070507203}   % 1st member
%%% Put other group member names here ..
\def\dissauthortwo{นายอัษฎาวุธ โหมดเทศ\qquad65070507228}   % 2nd member (optional)
\def\dissauthorthree{นายบุรินทร์ ราชกิจจา\qquad65070507235}   % 3rd member (optional)
\def\dissauthorfour{นายสหสวรรษ ศรีแจ่มใส\qquad65070507237}   % 3rd member (optional)


% The degree that you're persuing..
\def\dissdegree{Bachelor of Engineering} % Name of the degree
\def\dissdegreeabrev{B.Eng} % Abbreviation of the degree
\def\dissyear{2025}                   % Year of submission
\def\thaidissyear{2568}               % Year of submission (B.E.)

%%%%%%%%%%%%%%%%%%%%%%%%%%%%%%%%%%%%%%%%%%%%
% Your project and independent study committee..
%%%%%%%%%%%%%%%%%%%%%%%%%%%%%%%%%%%%%%%%%%%%
\def\dissadvisor{Assoc.Prof. ดร.ประพงษ์ ปรีชาประพาฬวงศ์ , Ph.D.}  % Advisor
%%% Leave it empty if you have no Co-advisor
\def\disscoadvisor{}  % Co-advisor
%\def\disscoadvisortwo{}  % Co-advisor 2 (if any)
\def\disscoadvisorthree{}  % Co-advisor 3 (You better be building space rocket or curing cancer at this point)
\def\disscommitteetwo{Asst.Prof. ผศ. ดร. สุธาทิพย์ มณีวงศ์วัฒนา, Ph.D.}  % 3rd committee member (optional)
\def\disscommitteethree{Asst.Prof. ดร. ณัฐชา เดชดำรง, Ph.D.}   % 4th committee member (optional) 
\def\disscommitteefour{}    % 5th committee member (optional) 

\def\worktype{Project} %%  Project or Independent study
\def\disscredit{3}   %% 3 credits or 6 credits


\def\fieldofstudy{Computer Engineering} 
\def\department{Computer Engineering} 
\def\faculty{Engineering}

\def\thaifieldofstudy{วิศวกรรมคอมพิวเตอร์} 
\def\thaidepartment{วิศวกรรมคอมพิวเตอร์} 
\def\thaifaculty{วิศวกรรมศาสตร์}
 
\def\appendixnames{Appendix} %%% Appendices or Appendix

\def\thaiworktype{ปริญญานิพนธ์} %  Project or research project % 
\def\thaidisstitleone{หัวข้อปริญญานิพนธ์บรรทัดแรก}
\def\thaidisstitletwo{หัวข้อปริญญานิพนธ์บรรทัดสอง}
\def\thaidissauthor{นายสมศักดิ์ คอมพิวเตอร์}
\def\thaidissauthortwo{นางสาวสมศรี คอมพิวเตอร์2} %Optional
\def\thaidissauthorthree{นางสาวสมปอง คอมพิวเตอร์3} %Optional

\def\thaidissadvisor{รศ.ดร.ที่ปรึกษา วิทยานิพนธ์}
%% Leave this empty if you have no co-advisor
\def\thaidisscoadvisor{รศ.ดร.ที่ปรึกษา วิทยานิพนธ์ร่วม} %Optional
\def\thaidisscoadvisortwo{}% Co-advisor 2 (if any)
\def\thaidisscoadvisorthree{} % Co-advisor 3 (You better be building space rocket or curing cancer at this point)
\def\thaidissdegree{วิศวกรรมศาสตรบัณฑิต}

% Change the line spacing here...
\linespread{1.15}

%%%%%%%%%%%%%%%%%%%%%%%%%%%%%%%%%%%%%%%%%%%%%%%%%%%%%%%%%%%%%%%%
% End of personal customization.  Do not modify from this part 
% to \begin{document} unless you know what you are doing...
%%%%%%%%%%%%%%%%%%%%%%%%%%%%%%%%%%%%%%%%%%%%%%%%%%%%%%%%%%%%%%%%


%%%%%%%%%%%% Dissertation style %%%%%%%%%%%
%\linespread{1.6} % Double-spaced  
%%\oddsidemargin    0.5in
%%\evensidemargin   0.5in
%%%%%%%%%%%%%%%%%%%%%%%%%%%%%%%%%%%%%%%%%%%
%\renewcommand{\subfigtopskip}{10pt}
%\renewcommand{\subfigbottomskip}{-5pt} 
%\renewcommand{\subfigcapskip}{-6pt} %vertical space between caption
%                                    %and figure.
%\renewcommand{\subfigcapmargin}{0pt}

\renewcommand{\topfraction}{0.85}
\renewcommand{\textfraction}{0.1}

\newtheorem{theorem}{Theorem}
\newtheorem{lemma}{Lemma}
\newtheorem{corollary}{Corollary}

\def\QED{\mbox{\rule[0pt]{1.5ex}{1.5ex}}}
\def\proof{\noindent\hspace{2em}{\itshape Proof: }}
\def\endproof{\hspace*{\fill}~\QED\par\endtrivlist\unskip}
%\newenvironment{proof}{{\sc Proof:}}{~\hfill \blacksquare}
%% The hyperref package redefines the \appendix. This one 
%% is from the dissertation.cls
%\def\appendix#1{\iffirstappendix \appendixcover \firstappendixfalse \fi \chapter{#1}}
%\renewcommand{\arraystretch}{0.8}
%%%%%%%%%%%%%%%%%%%%%%%%%%%%%%%%%%%%%%%%%%%%%%%%%%%%%%%%%%%%%%%%
%%%%%%%%%%%%%%%%%%%%%%%%%%%%%%%%%%%%%%%%%%%%%%%%%%%%%%%%%%%%%%%%

\usepackage{ragged2e}
\begin{document}

\pdfstringdefDisableCommands{%
\let\MakeUppercase\relax
}

\begin{center}
  \includegraphics[width=2.8cm]{logo02.jpg}
\end{center}
\vspace*{-1cm}

\maketitlepage
\makesignaturepage 

%%%%%%%%%%%%%%%%%%%%%%%%%%%%%%%%%%%%%%%%%%%%%%%%%%%%%%%%%%%%%%
%%%%%%%%%%%%%%%%%%%%%% English abstract %%%%%%%%%%%%%%%%%%%%%%%
%%%%%%%%%%%%%%%%%%%%%%%%%%%%%%%%%%%%%%%%%%%%%%%%%%%%%%%%%%%%%%
\abstract

In a multihop ad hoc network, the interference among nodes is
  reduced to maximize the throughput by using a smallest transmission
  range that still preserve the network connectivity. However, most
  existing works on transmission range control focus on the
  connectivity but lack of results on the throughput performance. This
  paper analyzes the per-node saturated throughput of an IEEE 802.11b
  multihop ad hoc network with a uniform transmission range. Compared
  to simulation, our model can accurately predict the per-node
  throughput.  The results show that the maximum achievable per-node
  throughput can be as low as 11\% of the channel capacity in a normal
  set of $\alpha$ operating parameters independent of node density. However, if
  the network connectivity is considered, the obtainable throughput
  will reduce by as many as 43\% of the maximum throughput. 

\begin{flushleft}
\begin{tabular*}{\textwidth}{@{}lp{0.8\textwidth}}
\textbf{Keywords}: & Multihop ad hoc networks / Topology control / Single-Hop Throughput
\end{tabular*}
\end{flushleft}
\endabstract

%%%%%%%%%%%%%%%%%%%%%%%%%%%%%%%%%%%%%%%%%%%%%%%%%%%%%%%%%%%%%%
%%%%%%%%%% Thai abstract here %%%%%%%%%%%%%%%%%%%%%%%%%%%%%%%%%
%%%%%%%%%%%%%%%%%%%%%%%%%%%%%%%%%%%%%%%%%%%%%%%%%%%%%%%%%%%%%%
% {\newfontfamily\thaifont{TH Sarabun New:script=thai}[Scale=1.3]
% \XeTeXlinebreaklocale "th_TH"	
% \thaifont
\thaiabstract

เซ็นเซอร์ เอ็กซ์เพรสรองรับคอนเซปต์สหัสวรรษเมจิก อิ่มแปร้ เฟรชชี่ ชาร์ปเช็งเม้งคลาสสิก แพตเทิร์น แอลมอนด์ เพลซว้อยก๊วน ซาร์ดีนซี้เนิร์สเซอรีอีสต์ สเตเดียมเพียบแปร้โอ้ยแคมปัส จัมโบ้ช็อตแมคเคอเรลอึ๋ม สตริง แมกกาซีนสตริงผ้าห่ม ฮัลโหล ยิม รอยัลตี้

\begin{flushleft}
\begin{tabular*}{\textwidth}{@{}lp{0.8\textwidth}}
 & \\

\textbf{คำสำคัญ}: & การชุบเคลือบด้วยไฟฟ้า / การชุบเคลือบผิวเหล็ก /  เคลือบผิวรังสี
\end{tabular*}
\end{flushleft}
\endabstract

%}

%%%%%%%%%%%%%%%%%%%%%%%%%%%%%%%%%%%%%%%%%%%%%%%%%%%%%%%%%%%%
%%%%%%%%%%%%%%%%%%%%%%% Acknowledgments %%%%%%%%%%%%%%%%%%%%
%%%%%%%%%%%%%%%%%%%%%%%%%%%%%%%%%%%%%%%%%%%%%%%%%%%%%%%%%%%%
\preface
ขอบคุณอาจารย์ที่ปรึกษา กรรมการ พ่อแม่พี่น้อง และเพื่อนๆ คนที่ช่วยให้งานสำเร็จ ตามต้องการ

%%%%%%%%%%%%%%%%%%%%%%%%%%%%%%%%%%%%%%%%%%%%%%%%%%%%%%%%%%%%%
%%%%%%%%%%%%%%%% ToC, List of figures/tables %%%%%%%%%%%%%%%%
%%%%%%%%%%%%%%%%%%%%%%%%%%%%%%%%%%%%%%%%%%%%%%%%%%%%%%%%%%%%%
% The three commands below automatically generate the table 
% of content, list of tables and list of figures
\tableofcontents                    
\listoftables
\listoffigures                      

%%%%%%%%%%%%%%%%%%%%%%%%%%%%%%%%%%%%%%%%%%%%%%%%%%%%%%%%%%%%%%
%%%%%%%%%%%%%%%%%%%%% List of symbols page %%%%%%%%%%%%%%%%%%%
%%%%%%%%%%%%%%%%%%%%%%%%%%%%%%%%%%%%%%%%%%%%%%%%%%%%%%%%%%%%%%
% You have to add this manually..
\listofsymbols
\begin{flushleft}
\begin{tabular}{@{}p{0.07\textwidth}p{0.7\textwidth}p{0.1\textwidth}}
\textbf{SYMBOL}  & & \textbf{UNIT} \\[0.2cm]
$\alpha$ & Test variable\hfill & m$^2$ \\
$\lambda$ & Interarival rate\hfill &  jobs/second\\
$\mu$ & Service rate\hfill & jobs/second\\
\end{tabular}
\end{flushleft}
%%%%%%%%%%%%%%%%%%%%%%%%%%%%%%%%%%%%%%%%%%%%%%%%%%%%%%%%%%%%%%
%%%%%%%%%%%%%%%%%%%%% List of vocabs & terms %%%%%%%%%%%%%%%%%
%%%%%%%%%%%%%%%%%%%%%%%%%%%%%%%%%%%%%%%%%%%%%%%%%%%%%%%%%%%%%%
% You also have to add this manually..
\listofvocab
\begin{flushleft}
\begin{tabular}{@{}p{1in}@{=\extracolsep{0.5in}}p{0.73\textwidth}}
Test &  Lorem ipsum dolor sit amet, consectetur adipiscing elit. Nullam non condimentum purus. Pellentesque sed augue sapien. In volutpat quis diam laoreet suscipit. Curabitur fringilla sem nisi, at condimentum lectus consequat vitae.\\
MANET & Mobile Ad Hoc Network 
\end{tabular}
\end{flushleft}

%\setlength{\parskip}{1.2mm}

%%%%%%%%%%%%%%%%%%%%%%%%%%%%%%%%%%%%%%%%%%%%%%%%%%%%%%%%%%%%%%%
%%%%%%%%%%%%%%%%%%%%%%%% Main body %%%%%%%%%%%%%%%%%%%%%%%%%%%%
%%%%%%%%%%%%%%%%%%%%%%%%%%%%%%%%%%%%%%%%%%%%%%%%%%%%%%%%%%%%%%%


\chapter{บทนำ}

\emph{หัวข้อต่าง ๆ ในแต่ละบทเป็นเพียงตัวอย่างเท่านั้น หัวข้อที่จะใส่ในแต่ละบทขึ้นอยู่กับโปรเจคของนักศึกษาและอาจารย์ที่ปรึกษา}

\section{ที่มาและความสำคัญ}

ตัวอย่างการใส่อ้างอิงที่มา -> \cite{hypersense} ถ้าต้องการใส่แหล่งอ้างอิงมากกว่า 1 ให้ทำดังนี้ -> \cite{hypersense,bworld} มนุษย์มีความสามารถในการประดิษฐ์คิดค้น มาตั้งแต่สมัยโบราณ ย้อนกลับไปตั้งแต่สมัยยุคปฏิวัติอุตสาหกรรม ที่มนุษย์ได้คิดค้นเครื่องจักรไอน้ำขึ้นมาแล้ว เพื่อเป็นเครื่องทุ่นแรงในการผลิตสิ่งต่างๆ กาลเวลาผ่านพลังไอน้ำก็แปรเปลี่ยนเป็นพลังงานไฟฟ้า จนต่อมาก็ได้มีสิ่งประดิษฐ์ที่พลิกประวัติศาสตร์โลกเกิดขึ้น นั่นก็คือเครื่องคอมพิวเตอร์ การมาของคอมพิวเตอร์นั่นช่วยให้เครื่องจักรสามารถควบคุมแบบอัตโนมัติได้ แม้คอมพิวเตอร์จะมีประโยชน์เป็นอย่างมาก แต่ก็ปฏิเสธไม่ได้ว่าบางอย่างการควบคุมโดย มนุษย์นั้นมีความจำเป็นมากกว่า ซึ่งในปัจจุบันการควบคุมคอมพิวเตอร์ของมนุษย์ ไม่ได้ใช้อวัยวะเพียงแค่มือสองมือ แต่ยังมีการนำอวัยวะอื่นภายในร่างกายมาใช้ควบคุมคอมพิวเตอร์ด้วย ยกตัวอย่างเช่น Amazon Alexa เป็นลำโพงที่เราสามารถออกคำสั่งเสียงเพื่อควบคุมการทำงานต่างๆ ไม่ว่าจะเป็น การตั้งเวลา, สร้างกิจกรรมในปฏิทิน, การแจ้งเตือน, การตรวจเช็คข่าวหรือแม้กระทั่งการสั่งการให้ เปิด-ปิด หลอดไฟภายในห้องได้ อีกทั้งยังมี Kinect Xbox ที่เป็นอุปกรณ์ที่ใช้ตรวจจับการเคลื่อนไหวแล้วนำไปควบคุมตัวละครภายในวีดีโอเกม จนทำให้เกิดความคิดที่จะใช้สมองควบคุมคอมพิวเตอร์โดยตรง โดยหวังผลให้เกิดประสิทธิภาพที่ดีขึ้นกว่าการใช้อวัยวะในการควบคุม จึงเป็นจุดเริ่มต้นของการจินตนาการการเคลื่อนไหว (Motor Imagery) ซึ่งเป็นการจินตนาการว่าเราต้องการจะทำอะไร โดยที่เราไม่ได้ทำสิ่งนั้นจริง เมื่อเราจินตนาการสมองของเราจะส่งสัญญาณคลื่นไฟฟ้าสมองออกมา ซึ่งสามารถตรวจวัดได้ด้วยเครื่องวัดสัญญาณไฟฟ้าสมอง (EEG) 
แต่ด้วยความยุ่งยากของอุปกรณ์เครื่องวัดสัญญาณคลื่นไฟฟ้าสมองและมีค่าใช้จ่ายที่ค่อนข้างสูง ทางกลุ่มเราจึงเล็งเห็นว่า อยากที่จะพัฒนาอุปกรณ์เครื่องวัดสัญญาณคลื่นไฟฟ้าสมอง (EEG) โดยมีการลดจำนวนขั้ววัดสัญญาณคลื่นไฟฟ้าสมองให้น้อยลง และมีการพัฒนาการแยกประเภทของสัญญาณให้ดีขึ้น เพื่อการทำงานและควบคุมได้หลากหลายรูปแบบขึ้น ตามอุปกรณ์เครื่องวัดสัญญาณคลื่นไฟฟ้าสมองที่เราใช้ หากผลงานเสร็จสมบูรณ์ จะช่วยให้ผู้คนสามารถเข้าถึงและใช้ง่ายอุปกรณ์เครื่องวัดสัญญาณคลื่นไฟฟ้าสมองได้ง่ายขึ้น จากการที่ความยุ่งยากและค่าใช้จ่ายที่ของอุปกรณ์ลดลง และสามารถนำไปประยุกต์ใช้ในการใช้งานต่างๆได้ เช่น การฟื้นฟูสมรรถภาพทางสมองสำหรับนักกีฬา, การฟื้นฟูสมรรถภาพในผู้ป่วยที่ได้รับผลกระทบจากโรคหลอดเลือดสมอง, การควบคุมอุปกรณ์ช่วยเหลือสำหรับผู้พิการ หรือการเล่นเกมส์ เป็นต้น
%
%\url{http://www.cpe.kmutt.ac.th}


วิธีการใส่ลิ้งค์จากเว็บไซต์ ->
\href{http://www.cpe.kmutt.ac.th} {http://www.cpe.kmutt.ac.th}

\cite{bworld}

\begin{figure}[!h]
\centering
\fbox{\includegraphics[width=10cm]{./chart-figure-5.png}}
\caption{This is the figure x1 ทดสอบ จาก \href{https://www.google.com} {https://www.google.com}}\label{fig:x1}
\end{figure}


Explain the motivations of your works.  
\begin{itemize}
\item   What are the problems you are addressing? 
\item  Why they are important?
\item  What are the limitations of existing approaches? 
\end{itemize}
You may combine this section with the background section.



\section{วัตถุประสงค์}

ระบุสิ่งท่ี่จะทำในโครงการ ซึ่งจะใช้สำหรับการประเมินว่าโครงงานทำสำเร็จหรือไม่ 

\section{ขอบเขตของโครงงาน}

Explain the scope of your works. 
\begin{itemize}
\item   What are the problems you are addressing? 
\item  Why they are important?
\item  What are the limitations of existing approaches? 
\end{itemize}

\section{ประโยชน์ที่คาดว่าจะได้่รับ}

โครงงานนี้จะเป็นประโยชน์กับใคร ยังไง ทั้งในเชิงรูปธรรมและนามธรรม ในปัจจุบันหรือในอนาคตถ้านำไป
ต่อยอด

\section{ตารางการดำเนินงาน}




%%%%%%%%%%%%%%%%%%%%%%%%%%%%%%%%%%%%%%%%%%%%%%%%%%%%%%%%%%%%
%%%%%%%%%%%%%%  Literature Review %%%%%%%%%%%%%%%%%%%%%%%%%%
%%%%%%%%%%%%%%%%%%%%%%%%%%%%%%%%%%%%%%%%%%%%%%%%%%%%%%%%%%%%
\chapter{ทฤษฎีความรู้และงานที่เกี่ยวข้อง}

\emph{หัวข้อต่าง ๆ ในแต่ละบทเป็นเพียงตัวอย่างเท่านั้น หัวข้อที่จะใส่ในแต่ละบทขึ้นอยู่กับโปรเจคของนักศึกษาและอาจารย์ที่ปรึกษา}

ตัวอย่างการใส่อ้างอิงที่มา -> \cite{hypersense} ถ้าต้องการใส่แหล่งอ้างอิงมากกว่า 1 ให้ทำดังนี้ -> \cite{hypersense,bworld} 
อธิบายทฤษฎี องค์ความรู้หลักที่ใช้ในงาน งานวิจัยที่นำมาใช้ในโครงงาน หรือเปรียบเทียบผลิตภัณฑ์ที่มีอยู่ในท้องตลาด\cite{bworld}
Explain theory, algorithms, protocols, or existing research works and tools related to your work. 


\section{ระบบแนะนำสินค้า}

\begin{table}[!h]
\caption{test table method1}\label{tbl:method1}
\begin{tabular}{c|c|l|rr} \hline\hline
Center & Center & left aligned & Right & Right aligned \\ \hline\hline
Center & Center & left aligned & Right & Right aligned \\ \hline
Center & Center & left aligned & Right & Right aligned \\ 
Center & Center & left aligned & Right & Right aligned \\ \hline
Center & Center & left aligned & Right & Right aligned \\ \hline\hline
\end{tabular}
\end{table}


\section{อัลกอริทึมในการประมวลผลข้อความ}
\subsection{อัลกอริทึม I}

% Can define this in the preamble..
You can place the figure and refer to it as รูปที่~\ref{fig:model2}.
The figure and table numbering will be run and updated automatically when you add/remove tables/figures from the document.

\begin{figure}[!h]\centering
\setlength{\fboxrule}{0.2mm} % can define this in the preamble
\setlength{\fboxsep}{1cm}
\fbox{\includegraphics[width=5cm]{./model2.pdf}}
\caption{The network model}\label{fig:model2}
\end{figure}

 
\subsection{อัลกอริทึม II}
Add more subsections as you want.
\subsubsection{ขั้นตอนที่ 1}
\subsubsection{ขั้นตอนที่ 2}
Latex Format นี้รองรับหัวข้อย่อยถึงแค่ระดับ 4 นี้เท่านั้น ไม่แนะนำให้แบ่งหัวข้อย่อยไปมากกว่านี้ เช่น 2.2.2.2.1 , 2.2.2.2.2

\section{เครื่องมือที่ใช้ในการพัฒนา}

%%%%%%%%%%%%%%%%%%%%%%%%%%%%%%%%%%%%%%%%%%%%%%%%%%%%%55
%%%%%%%%%%%%%%%%%%%%%%%%%%%%%%%%%%%%%%%%%%%%%%%%%%%%%
%%%%%%%%%%%%%%%%%%%%%%%%%%%%%%%%%%%%%%%%%%%%%%%%%%%%%
\chapter{วิธีการดำเนินงาน}

\emph{หัวข้อต่าง ๆ ในแต่ละบทเป็นเพียงตัวอย่างเท่านั้น หัวข้อที่จะใส่ในแต่ละบทขึ้นอยู่กับโปรเจคของนักศึกษาและอาจารย์ที่ปรึกษา}


ตัวอย่างการใส่อ้างอิงที่มา -> \cite{hypersense} ถ้าต้องการใส่แหล่งอ้างอิงมากกว่า 1 ให้ทำดังนี้ -> \cite{hypersense,bworld} 
Explain the design (how you plan to implement your work) of your project. Adjust the section titles below to suit the types of your work. Detailed physical design like circuits and source codes should be placed in the appendix.

\section{การสำรวจความต้องการกับผู้ใช้}

\section{ความสามารถของระบบ}
\subsection{Use Case Diagram}
\begin{figure}[!h]\centering
 \setlength{\fboxrule}{0.2mm} 
\setlength{\fboxsep}{5pt}  
\fbox{\includegraphics[width=0.7\textwidth]{./Diagram/FastPass Diagram-Ues Case _การจอง.pdf}}
\caption{Ues Case Diagram (ระบบจอดรถ)}\label{fig:FastPass Diagram-Ues Case _การจอง}
\end{figure}

\clearpage
\begin{figure}[!h]\centering
 \setlength{\fboxrule}{0.2mm} 
\setlength{\fboxsep}{5pt}  
\fbox{\includegraphics[width=0.7\textwidth]{./Diagram/FastPass Diagram-Use Case _ระบบจัดการส่วนกลาง.pdf}} 
\caption{Ues Case Diagram (ระบบจัดการส่วนกลาง)}\label{fig:model2}
\end{figure}

\subsection{Use Case Narrative}

%%%%%%%%%%%%%%%%%%%%%%%%%%%%%%% การนำรถเข้าจอด %%%%%%%%%%%%%%%%%%%%%%%%%%%%%%%
\subsubsection{การนำรถเข้าจอด}

\begin{table}[h!]
\centering
\renewcommand{\arraystretch}{1.3} % เพิ่มความสูงบรรทัดให้ดูไม่อึดอัด
\begin{tabular}{p{3.5cm} p{11cm}} % กำหนดความกว้าง: คอลัมน์ซ้าย 3.5cm, ขวา 11cm
    \hline
    Use Case Name & การนำรถเข้าจอด (Park Vehicle) \\
    \hline
    Actor & บุคลากร (Staff) / ผู้มาติดต่อ (Visitor) / บุคคลภายนอก(Guest) \\
    \hline
    Goal & เพื่อนำรถผ่านไม้กั้นและเข้าจอดในช่องจอดที่กำหนด \\
    \hline
    Precondition & ผู้ใช้งานต้องมีการจองในระบบ หรือได้รับสิทธิ์ Visitor เรียบร้อยแล้ว \\
    \hline
    Main success scenario & 
    \vspace{-0.5em} 
    \begin{enumerate}[leftmargin=*, nosep, after=\vspace{0.5em}]
        \item ผู้ใช้งานขับรถมาถึงจุดทางเข้า (Barrier Gate)
        \item ผู้ใช้งานแสดงหลักฐาน (เช่น กดรับ QR Code ที่ตู้ Kiosk หรือระบบอ่านป้ายทะเบียนอัตโนมัติ)
        \item ระบบตรวจสอบสิทธิ์การเข้าใช้งานจากฐานข้อมูล
        \item ระบบยืนยันสิทธิ์ถูกต้องและบันทึกเวลาเข้า
        \item ระบบสั่งเปิดไม้กั้นทางเข้า
        \item ผู้ใช้งานขับรถผ่านไม้กั้นเข้าไปยังลานจอด
        \item ระบบตรวจจับว่ารถผ่านไปแล้วและสั่งปิดไม้กั้น
        \item ระบบอัปเดตสถานะช่องจอดเป็น "ไม่ว่าง"
    \end{enumerate} 
    \\
    \hline
    Extensions (a) & 
    \vspace{-0.5em}
    % start=3 คือเริ่มที่เลข 3 ตามขั้นตอนที่มีปัญหาใน Main flow
    \begin{enumerate}[label=\arabic*a., start=3, leftmargin=*, nosep, after=\vspace{0.5em}]
        \item ระบบตรวจสอบไม่พบข้อมูลการจอง หรือสิทธิ์ไม่ถูกต้อง
        \item ระบบแสดงข้อความแจ้งเตือน "ไม่พบสิทธิ์การเข้าใช้งาน" ที่หน้าจอ
        \item ไม้กั้นยังคงปิดอยู่
        \item กลับไปที่ขั้นตอนที่ 2 (เพื่อให้ลองใหม่ หรือติดต่อเจ้าหน้าที่)
    \end{enumerate}
    \\
    \hline
    Postcondition & รถของผู้ใช้งานเข้าจอดในพื้นที่ และสถานะช่องจอดในระบบถูกอัปเดตเรียบร้อยแล้ว \\
    \hline
\end{tabular}
\label{tab:usecase_parking_entry}
\end{table}

%%%%%%%%%%%%%%%%%%%%%%%%%%%%%%% การจองที่จอดรถ %%%%%%%%%%%%%%%%%%%%%%%%%%%%%%%
\clearpage
\subsubsection{การจองที่จอดรถ}
\begin{table}[h!]
\centering
\renewcommand{\arraystretch}{1.3} 
\begin{tabular}{p{3.5cm} p{11cm}} 
    \hline
    Use Case Name & จองที่จอดรถ \\
    \hline
    Actor & บุคลากร (Staff) / ผู้มาติดต่อ (Visitor) \\
    \hline
    Goal & เพื่อทำการจองช่องจอดรถล่วงหน้าสำหรับการใช้งาน \\
    \hline
    Precondition & ผู้ใช้งานต้องเข้าสู่ระบบ (Login) เรียบร้อยแล้ว และเป็นบุคลากร (Staff) / ผู้มาติดต่อ (Visitor) \\
    \hline
    Main success scenario & 
    \vspace{-0.5em} 
    \begin{enumerate}[leftmargin=*, nosep, after=\vspace{0.5em}]
        \item ผู้ใช้งานใช้งานเมนู "แผนที่"
        \item ระบบแสดงแผนที่และสถานะช่องจอด
        \item ผู้ใช้งานเลือกสถานที่ต้องการ และระบุรายละเอียด ชั้น (Floor), โซน (Zone)
        \item ผู้ใช้กดปุ่ม "จอง"  ระบบแสดง modal ให้เลือก วันที่-เวลาที่ต้องการเลือก
        \item ผู้ใช้กดปุ่ม "ตรวจสอบการจองสิทธิ"
	\item ระบบเรียกฟังก์ชัน "แสดงรายละเอียดการจอง" เพื่อแสดงรายละเอียดการจองให้ผู้ใช้เห็นทันที
        \item ผู้ใช้งานยืนยันการจอง
        \item ระบบบันทึกข้อมูลการจองลงในฐานข้อมูล

    \end{enumerate} 
    \\
    \hline
    Extensions (a) & 
    \vspace{-0.5em}
    \begin{enumerate}[label=\arabic*a., start=4, leftmargin=*, nosep, after=\vspace{0.5em}]
        \item ช่องจอดที่เลือกถูกจองไว้แล้ว หรือไม่ว่างในช่วงเวลานั้น
        \item ระบบแจ้งเตือน "ไม่สามารถเลือกที่จอดได้เนื่องจาก เต็ม!"
        \item ระบบจะให้ผู้ใช้งานเลือกวันที่และเวลาการจองใหม่
    \end{enumerate}
    \\
    \hline
    Postcondition & ข้อมูลการจองถูกบันทึก และสถานะของช่องจอดในช่วงเวลานั้นถูกเปลี่ยนเป็น "อยู่ระหว่างการจอง" \\
    \hline
\end{tabular}
\label{tab:usecase_booking}
\end{table}

%%%%%%%%%%%%%%%%%%%%%%%%%%%%%%% บันทึกส่วนลดค่าจอดรถ %%%%%%%%%%%%%%%%%%%%%%%%%%%%%%%

% ดันหัวข้อลงไปหน้าใหม่
%\clearpage 

\subsubsection{บันทึกส่วนลดค่าจอดรถ}

\begin{table}[h!]
\centering
\renewcommand{\arraystretch}{1.3} % เพิ่มความสูงบรรทัดให้ดูไม่อึดอัด
\begin{tabular}{p{3.5cm} p{11cm}} % กำหนดความกว้าง: คอลัมน์ซ้าย 3.5cm, ขวา 11cm
    \hline
    Use Case Name & บันทึกส่วนลดค่าจอดรถ (Record Parking Discount) \\
    \hline
    Actor & บุคลากร (Staff) / ผู้มาติดต่อ (Visitor) \\
    \hline
    Goal & เพื่อบันทึกสิทธิ์ส่วนลดค่าจอดรถจากการใช้บริการร้านค้าหรือโปรโมชันต่างๆ \\
    \hline
    Precondition & ผู้ใช้งานต้องมีสถานะการจอดรถอยู่ในระบบ Check-in แล้ว และกำลังเข้าร่วมบริการร้านค้าหรือโปรโมชันต่างๆ \\
    \hline
    Main success scenario & 
    \vspace{-0.5em} 
    \begin{enumerate}[leftmargin=*, nosep, after=\vspace{0.5em}]
        \item ผู้ใช้งานเลือกเมนู "ส่วนลดค่าจอดรถ" หรือสแกน QR Code ส่วนลดจากใบเสร็จ
        \item ผู้ใช้งานกรอกรหัสส่วนลด หรือให้ศูนย์บริการแสดง QR Code ของลูกค้าที่อยู่ในเว็ปแอป
        \item ระบบตรวจสอบความถูกต้อง
        \item ระบบยืนยันว่าส่วนลดสามารถใช้งานได้
        \item ระบบเรียกใช้ "จำนวนชั่วโมงในการจอดฟรี" เพื่อคำนวณและเพิ่มสิทธิ์ส่วนลดเป็นเวลาจอดฟรีให้กับผู้ใช้งาน
        \item ระบบแสดงผลจำนวนชั่วโมงที่ได้รับฟรี และเวลาการจอดรถฟรีที่เหลือ
        \item ระบบบันทึกข้อมูลส่วนลดลงในประวัติการจอดรถ
    \end{enumerate} 
    \\
    \hline
    Extensions (a) & 
    \vspace{-0.5em}
    \begin{enumerate}[label=\arabic*a., start=3, leftmargin=*, nosep, after=\vspace{0.5em}]
        \item รหัสส่วนลดไม่ถูกต้อง หมดอายุ หรือถูกใช้งานไปแล้ว
        \item ระบบแจ้งเตือน "ไม่สามารถใช้ส่วนลดได้ เนื่องจากสิทธิ๋ถูกใช้ไปเรียบร้อยแล้ว พร้อมระบุเวลาที่โดนใช้ไป DD-MM-YYYY HH:MM"
    \end{enumerate}
    \\
    \hline
    Postcondition & สิทธิ์ส่วนลดถูกนำไปคำนวณค่าจอดรถ และสถานะค่าบริการได้รับการอัปเดต \\
    \hline
\end{tabular}
\label{tab:usecase_discount}
\end{table}

%%%%%%%%%%%%%%%%%%%%%%%%%%%%%%% ดูสถานที่จอดรถ %%%%%%%%%%%%%%%%%%%%%%%%%%%%%%%
\clearpage 

\subsubsection{ดูสถานที่จอดรถ}

\begin{table}[h!] 
\centering
\renewcommand{\arraystretch}{1.3} % เพิ่มความสูงบรรทัดให้ดูไม่อึดอัด
\begin{tabular}{p{3.5cm} p{11cm}} % กำหนดความกว้าง: คอลัมน์ซ้าย 3.5cm, ขวา 11cm
    \hline
    Use Case Name & ดูสถานที่จอดรถ (View Parking Status) \\
    \hline
    Actor & บุคลากร (Staff) / ผู้มาติดต่อ (Visitor) / บุคคลภายนอก(Guest) \\
    \hline
    Goal & เพื่อดูสถานะและตำแหน่งของช่องจอดรถในปัจจุบัน \\
    \hline
    Precondition & ผู้ใช้งานเข้าถึงระบบผ่านหน้าเว็บแอปพลิเคชัน ในส่วนนี้ไม่จำเป็นต้อง Login สำหรับ Guest \\
    \hline
    Main success scenario & 
    \vspace{-0.5em} 
    \begin{enumerate}[leftmargin=*, nosep, after=\vspace{0.5em}]
        \item ผู้ใช้งานเข้ามาระบบจะแสดงแผนที่ของลานจอดรถ
        \item ระบบแสดงแผนที่ของลานจอดรถจอดรถทั้งหมด
        \item ผู้ใช้งานเลือกโซนหรือชั้นที่ต้องการดู
        \item ระบบเรียก "รายงานจำนวนช่องจอดรถ" เพื่อดึงข้อมูล Real-time ของจำนวนช่องจอดที่ว่างในโซนที่ผู้ใช้เลือก
        \item ระบบแสดงผลกราฟิกแผนที่ พร้อมระบุสีสถานะและจำนวนช่องว่างคงเหลือ
    \end{enumerate} 
    \\
    \hline
    Extensions (a) & 
    \vspace{-0.5em}
    \begin{enumerate}[label=\arabic*a., start=4, leftmargin=*, nosep, after=\vspace{0.5em}]
        \item ระบบไม่สามารถเชื่อมต่อกับเซนเซอร์หรือฐานข้อมูลสถานะได้
        \item ระบบแสดงข้อความ "ไม่สามารถแสดงสถานะล่าสุดได้ (Offline)"
        \item ระบบแสดงข้อมูลล่าสุดที่แคชไว้พร้อมระบุเวลาอัปเดต
    \end{enumerate}
    \\
    \hline
    Postcondition & ผู้ใช้งานได้รับทราบข้อมูลสถานะที่จอดรถเพื่อประกอบการตัดสินใจในการเลือกช่องจอดรถ \\
    \hline
\end{tabular}
\label{tab:usecase_view_parking}
\end{table}

%%%%%%%%%%%%%%%%%%%%%%%%%%%%%%% จัดการช่องจอดรถรายการโปรด %%%%%%%%%%%%%%%%%%%%%%%%%%%%%%%
\subsubsection{จัดการช่องจอดรถรายการโปรด}

\begin{table}[h!] 
\centering
\renewcommand{\arraystretch}{1.3} % เพิ่มความสูงบรรทัดให้ดูไม่อึดอัด
\begin{tabular}{p{3.5cm} p{11cm}} % กำหนดความกว้าง: คอลัมน์ซ้าย 3.5cm, ขวา 11cm
    \hline
    Use Case Name & จัดการช่องจอดรถรายการโปรด (Manage Favorite Spots) \\
    \hline
    Actor & บุคลากร (Staff) / ผู้มาติดต่อ (Visitor) / บุคคลภายนอก(Guest) \\
    \hline
    Goal & เพื่อเพิ่ม ลบ หรือแก้ไขรายการช่องจอดรถที่ใช้บ่อย เพื่อความสะดวกรวดเร็วในการจอง \\
    \hline
    Precondition & ผู้ใช้งานต้องเข้าสู่ระบบเรียบร้อยแล้ว \\
    \hline
    Main success scenario & 
    \vspace{-0.5em} 
    \begin{enumerate}[leftmargin=*, nosep, after=\vspace{0.5em}]
        \item ผู้ใช้งานเลือกเมนู "บันทึกแล้ว"
        \item ระบบแสดงรายชื่อช่องจอดที่บันทึกไว้ (ถ้ามี)
        \item ผู้ใช้งานกดปุ่ม "เพิ่มรายการบันทึก" จากช่องจอดที่เลือก หรือกด "ลบ" รายการเดิม
        \item ระบบบันทึกการเปลี่ยนแปลงลงในฐานข้อมูลส่วนตัวของผู้ใช้
    \end{enumerate} 
    \\
    \hline
    Extensions (a) & 
    \vspace{-0.5em}
    \begin{enumerate}[label=\arabic*a., start=4, leftmargin=*, nosep, after=\vspace{0.5em}]
        \item กรณีเพิ่มช่องจอดที่ซ้ำกับที่มีอยู่แล้ว
        \item ระบบแจ้งเตือน "ยกเลิกรายการบันทึก"
        \item ระบบยกเลิกการเพิ่ม
    \end{enumerate}
    \\
    \hline
    Postcondition & รายการช่องจอดรถโปรดของผู้ใช้งานได้รับการอัปเดต บันทึกเป็นรายการบันทึก / เลิกบันทึกเป็นรายการบันทึก \\
    \hline
\end{tabular}
\label{tab:usecase_manage_favorites}
\end{table}

%%%%%%%%%%%%%%%%%%%%%%%%%%%%%%% ดูรายการแจ้งเตือน %%%%%%%%%%%%%%%%%%%%%%%%%%%%%%%

\clearpage 

\subsubsection{ดูรายการแจ้งเตือน}

\begin{table}[h!] 
\centering
\renewcommand{\arraystretch}{1.3} % เพิ่มความสูงบรรทัดให้ดูไม่อึดอัด
\begin{tabular}{p{3.5cm} p{11cm}} % กำหนดความกว้าง: คอลัมน์ซ้าย 3.5cm, ขวา 11cm
    \hline
    Use Case Name & ดูรายการแจ้งเตือน (View Notifications) \\
    \hline
    Actor & บุคลากร (Staff) / ผู้มาติดต่อ (Visitor) \\
    \hline
    Goal & ผู้ใช้จะได้รับการแจ้งเตือนต่างๆ เช่น สถานะการจอง หรือเตือนเวลาจอดเมื่อใกล้หมดเวลา \\
    \hline
    Precondition & ผู้ใช้งานต้องเข้าสู่ระบบและทำรายการจองเรียบร้อยแล้ว \\
    \hline
    Main success scenario & 
    \vspace{-0.5em} 
    \begin{enumerate}[leftmargin=*, nosep, after=\vspace{0.5em}]
        \item ผู้ใช้งานกดที่ไอคอน "กระดิ่งแจ้งเตือน" บนหน้าจอหลัก
        \item ระบบแสดงรายการแจ้งเตือนของผู้ใช้งาน
        \item ระบบแสดงรายการแจ้งเตือน โดยเรียงจากใหม่สุดไปเก่าสุด
        \item ผู้ใช้งานเลือกกดดูรายละเอียดของรายการที่ต้องการ
        \item ระบบแสดงรายละเอียดของแจ้งเตือนนั้นๆอย่างครบถ้วน
        \item ระบบเปลี่ยนสถานะของรายการนั้นเป็น "อ่านแล้ว" (กระดิ่งแจ้งเตือนจะไม่แสดงปุ่มสีแดงหากผู้ใช้อ่านทุกรายการแจ้งเตือนแล้ว)
    \end{enumerate} 
    \\
    \hline
    Extensions (a) & 
    \vspace{-0.5em}
    \begin{enumerate}[label=\arabic*a., start=2, leftmargin=*, nosep, after=\vspace{0.5em}]
        \item กรณีไม่มีรายการแจ้งเตือนเลย
        \item ระบบแสดงข้อความ "กระดิ่งแจ้งเตือน" (ไม่แสดงปุ่มสีแดง)
    \end{enumerate}
    \vspace{-0.5em}
    \begin{enumerate}[label=\arabic*b., start=4, leftmargin=*, nosep, after=\vspace{0.5em}]
        \item ผู้ใช้งานต้องการลบการแจ้งเตือน
        \item ผู้ใช้งานปัดรายการหรือกดปุ่มลบ
        \item ระบบลบรายการนั้นออกจากรายการแสดงผล
    \end{enumerate}
    \\
    \hline
    Postcondition & ผู้ใช้งานได้รับทราบข้อมูลข่าวสารหรือสถานะล่าสุด และสถานะการอ่านถูกอัปเดต \\
    \hline
\end{tabular}
\label{tab:usecase_view_notifications}
\end{table}

%%%%%%%%%%%%%%%%%%%%%%%%%%%%%%% ดูรายงาน Report ของลานจอดรถ %%%%%%%%%%%%%%%%%%%%%%%%%%%%%%%

% ดันหัวข้อลงไปหน้าใหม่
\clearpage 

\subsubsection{ดูรายงาน Report ของลานจอดรถ}

\begin{table}[h!] 
\centering
\renewcommand{\arraystretch}{1.3} % เพิ่มความสูงบรรทัดให้ดูไม่อึดอัด
\begin{tabular}{p{3.5cm} p{11cm}} % กำหนดความกว้าง: คอลัมน์ซ้าย 3.5cm, ขวา 11cm
    \hline
    Use Case Name & ดูรายงาน Report ของลานจอดรถ (View Parking Report) \\
    \hline
    Actor & Admin (ผู้ดูแลระบบ) \\
    \hline
    Goal & เพื่อดูภาพรวมสถิติ, รายได้, และสถานะการใช้งานของลานจอดรถสำหรับการวิเคราะห์และบริหารจัดการ \\
    \hline
    Precondition & Admin ต้องเข้าสู่ระบบเรียบร้อยแล้ว \\
    \hline
    Main success scenario & 
    \vspace{-0.5em} 
    \begin{enumerate}[leftmargin=*, nosep, after=\vspace{0.5em}]
        \item Admin เลือกเมนู "หน้าหลัก" หรือ "รายงานและวิเคราะห์"
        \item ระบบดึงข้อมูลสรุปทางสถิติจากฐานข้อมูล
        \item ระบบแสดงผล  "สรุปข้อมูลสำคัญ" ได้แก่
            \begin{itemize}[leftmargin=1.5em, nosep]
                \item จำนวนลานจอดรถทั้งหมดและที่เพิ่มใหม่
                \item ที่ว่างขณะนี้ (จำนวนคัน) เทียบกับความจุทั้งหมด
                \item อัตราการใช้งาน (\%) และแนวโน้มเทียบกับเมื่อวาน
                \item รายได้วันนี้ (บาท) และแนวโน้มเทียบกับเมื่อวาน
            \end{itemize}
        \item ระบบแสดง "กราฟและแผนภูมิ"
            \begin{itemize}[leftmargin=1.5em, nosep]
                \item กราฟวงกลมแสดงสัดส่วนประเภทรถ (รถยนต์, มอเตอร์ไซค์, EV)
                \item กราฟแท่งแสดงแนวโน้มการใช้งานรายวัน/รายเดือน
            \end{itemize}
        \item ระบบแสดง "สถานะเรียลไทม์" ของแต่ละลานจอด (เช่น ตึก S2 ว่างกี่ช่อง, ตึก N18 เต็มหรือไม่)
        \item Admin สามารถใช้งาน "ตัวกรอง" (Filter) เพื่อเลือกดูข้อมูลตามวันที่ (dd/mm/yyyy), ประเภทพาหนะ หรือสถานะ
        \item ระบบอัปเดตข้อมูลบนหน้าจอตามเงื่อนไขที่ Admin เลือก
    \end{enumerate} 
    \\
    \hline
    Extensions (a) & 
    \vspace{-0.5em}
    \begin{enumerate}[label=\arabic*a., start=6, leftmargin=*, nosep, after=\vspace{0.5em}]
        \item กรณีเลือกช่วงเวลาที่ไม่มีข้อมูลการใช้งาน
        \item ระบบแสดงกราฟว่างเปล่า และแจ้งเตือนว่า "ไม่พบข้อมูลในช่วงเวลาดังกล่าว"
    \end{enumerate}
    \vspace{-0.5em}
    \begin{enumerate}[label=\arabic*b., start=2, leftmargin=*, nosep, after=\vspace{0.5em}]
        \item ระบบฐานข้อมูลสถิติขัดข้อง
        \item ระบบแสดงข้อความ "Failed to load dashboard data"
    \end{enumerate}
    \\
    \hline
    Postcondition & Admin ได้รับทราบข้อมูลสรุปผลการดำเนินงานเพื่อนำไปใช้ในการตัดสินใจในอนาคต \\
    \hline
\end{tabular}
\label{tab:usecase_view_report}
\end{table}

%%%%%%%%%%%%%%%%%%%%%%%%%%%%%%% กำหนดค่าระบบจอดรถ %%%%%%%%%%%%%%%%%%%%%%%%%%%%%%%
% ดันหัวข้อลงไปหน้าใหม่
\clearpage 

\subsubsection{กำหนดค่าระบบจอดรถ}

\begin{table}[h!] 
\centering
\renewcommand{\arraystretch}{1.3} % เพิ่มความสูงบรรทัดให้ดูไม่อึดอัด
\begin{tabular}{p{3.5cm} p{11cm}} % กำหนดความกว้าง: คอลัมน์ซ้าย 3.5cm, ขวา 11cm
    \hline
    Use Case Name & กำหนดค่าระบบจอดรถ (Configure Parking System Settings) \\
    \hline
    Actor & Admin (ผู้ดูแลระบบ) \\
    \hline
    Goal & เพื่อตั้งค่าพารามิเตอร์ต่างๆ ของลานจอดรถ เช่น อัตราค่าบริการ, เวลาเปิด-ปิด และสถานะการเปิดใช้งานของช่องจอดต่างๆ \\
    \hline
    Precondition & Admin ต้องเข้าสู่ระบบเรียบร้อยแล้ว \\
    \hline
    Main success scenario & 
    \vspace{-0.5em} 
    \begin{enumerate}[leftmargin=*, nosep, after=\vspace{0.5em}]
        \item Admin เลือกเมนู "จัดการข้อมูลที่จอดรถ" จากแถบเมนูด้านซ้าย
        \item ระบบแสดงรายการลานจอดรถทั้งหมด (เช่น ลานจอดรถ 14 ชั้น S2, อาคาร N16) พร้อมสถานะปัจจุบัน, ราคา, และความจุ
        \item Admin เลือกกดปุ่ม "แก้ไข" (Edit) ที่รายการที่ต้องการปรับปรุง หรือกด "เพิ่มสถานที่ใหม่"
        \item ระบบแสดงฟอร์มสำหรับกำหนดค่า:
            \begin{itemize}[leftmargin=1.5em, nosep]
                \item \textbf{ข้อมูลทั่วไป:} ชื่อสถานที่, ที่อยู่
                \item \textbf{การใช้งาน:} ประเภทพาหนะที่รองรับ (รถยนต์, EV, มอเตอร์ไซค์)
                \item \textbf{ความจุ (Capacity):} จำนวนช่องจอดสูงสุด
                \item \textbf{ค่าบริการ:} ราคาต่อชั่วโมง (เช่น 10 บาท/ชม.)
                \item \textbf{เวลาทำการ:} เวลาเปิด-ปิด (เช่น 08:00 - 20:00 น.)
            \end{itemize}
        \item Admin ทำการแก้ไขข้อมูลที่ต้องการเปลี่ยนแปลง
        \item Admin กดปุ่ม "บันทึก"
        \item ระบบตรวจสอบความถูกต้องของข้อมูล (Validation) และบันทึกลลงฐานข้อมูล
        \item ระบบแสดงข้อความแจ้งเตือน "บันทึกข้อมูลสำเร็จ" และกลับสู่หน้ารายการ
    \end{enumerate} 
    \\
    \hline
    Extensions (a) & 
    \vspace{-0.5em}
    \begin{enumerate}[label=\arabic*a., start=3, leftmargin=*, nosep, after=\vspace{0.5em}]
        \item Admin ต้องการปิดปรับปรุงลานจอดชั่วคราว
        \item Admin กดปุ่ม "ปิด" หรือเปลี่ยนสถานะเป็น "กำลังจะปิด/ปิดใช้งาน"
        \item ระบบอัปเดตสถานะใน Real-time Dashboard เพื่อไม่ให้ User จองเข้ามาได้
    \end{enumerate}
    \\
    \hline
    Postcondition & ค่าการตั้งค่าใหม่ (เช่น ราคาใหม่, สถานะเปิด/ปิด) ถูกนำไปใช้ในระบบทันที \\
    \hline
\end{tabular}
\label{tab:usecase_config_system}
\end{table}

%%%%%%%%%%%%%%%%%%%%%%%%%%%%%%%%%%%%%%%%%%%%%%%%%%%%%%%%%%%%%%
%%%%%%%%%%%%%%%%%%%% Experiments %%%%%%%%%%%%%%%%%%%%%%%%%%%%%
%%%%%%%%%%%%%%%%%%%%%%%%%%%%%%%%%%%%%%%%%%%%%%%%%%%%%%%%%%%%%%%
\chapter{ผลการดำเนินงาน}


\emph{หัวข้อต่าง ๆ ในแต่ละบทเป็นเพียงตัวอย่างเท่านั้น หัวข้อที่จะใส่ในแต่ละบทขึ้นอยู่กับโปรเจคของนักศึกษาและอาจารย์ที่ปรึกษา}


ตัวอย่างการใส่อ้างอิงที่มา -> \cite{hypersense} ถ้าต้องการใส่แหล่งอ้างอิงมากกว่า 1 ให้ทำดังนี้ -> \cite{hypersense,bworld} 

You can title this chapter as \textbf{Preliminary Results} ผลการดำเนินงานเบื้องต้น or \textbf{Work Progress} ความก้าวหน้าโครงงาน for the progress reports. Present implementation or experimental results here and discuss them.
ใส่เฉพาะหัวข้อที่เกี่ยวข้องกับงานที่ทำ 

\section{ประสิทฺธิภาพการทำงานของระบบ} 
\section{ความพึงพอใจการใช้งาน}
\section{การวิเคราะห์ข้อมูลและผลการทดลอง}

%%%%%%%%%%%%%%%%%%%%%%%%%%%%%%%%%%%%%%%%%%%%%%%%%%%%%%%%%%%%%%%
%%%%%%%%%%%%%%%%%%%% Conclusions %%%%%%%%%%%%%%%%%%%%%%%%%%%%%
%%%%%%%%%%%%%%%%%%%%%%%%%%%%%%%%%%%%%%%%%%%%%%%%%%%%%%%%%%%%%%%
\chapter{บทสรุป}


\emph{หัวข้อต่าง ๆ ในแต่ละบทเป็นเพียงตัวอย่างเท่านั้น หัวข้อที่จะใส่ในแต่ละบทขึ้นอยู่กับโปรเจคของนักศึกษาและอาจารย์ที่ปรึกษา}



This chapter is optional for proposal and progress reports but 
is required for the final report.

\section{สรุปผลโครงงาน}
สรุปว่าโครงงานบรรลุตามวัตถุประสงค์ที่ตั้งไว้หรือไม่ อย่างไร 

\section{ปัญหาที่พบและการแก้ไข}
State your problems and how you fixed them.

\section{ข้อจำกัดและข้อเสนอแนะ}
ข้อจำกัดของโครงงาน What could be done in the future to make your projects better.

%%%%%%%%%%%%%%%%%%%%%%%%%%%%%%%%%%%%%%%%%%%%%%%%%%%%%%%%%%%%%%%
%%%%%%%%%%%%%%%%%%%% Bibliography %%%%%%%%%%%%%%%%%%%%%%%%%%%%%
%%%%%%%%%%%%%%%%%%%%%%%%%%%%%%%%%%%%%%%%%%%%%%%%%%%%%%%%%%%%%%%

%%%% Comment this in your report to show only references you have
%%%% cited. Otherwise, all the references below will be shown.
%\nocite{*}
%% Use the kmutt.bst for bibtex bibliography style 
%% You must have cpe.bib and string.bib in your current directory.
%% You may go to file .bbl to manually edit the bib items.

% Sept, 2021 by Thanin
% improve url breaks to prevent unnecessary big white spaces in some cases
\makeatletter
\g@addto@macro{\UrlBreaks}{\UrlOrds}
\makeatother
% 

\bibliographystyle{kmutt}
\bibliography{string,cpe}

%%%%%%%%%%%%%%%%%%%%%%%%%%%%%%%%%%%%%%%%%%%%%%%%%%%%%%%%%%%%%%%
%%%%%%%%%%%%%%%%%%%%%%%% Appendix %%%%%%%%%%%%%%%%%%%%%%%%%%%%%
%%%%%%%%%%%%%%%%%%%%%%%%%%%%%%%%%%%%%%%%%%%%%%%%%%%%%%%%%%%%%%%
\appendix{ชื่อภาคผนวกที่ 1}
\noindent{\large\bf ใส่หัวข้อตามความเหมาะสม} \\

This is where you put hardware circuit diagrams, detailed experimental data in tables or source codes, etc.. \\ \bigskip


 \begin{figure}[!h]
\caption{This is the figure x11 ทดสอบ จาก \href{https://www.google.com} {https://www.google.com}}\label{fig:x1}
\end{figure}


This appendix describes two static allocation methods for fGn (or fBm)
traffic. Here, $\lambda$ and $C$ are respectively the traffic arrival
rate and the service rate per dimensionless time step. Their unit are
converted to a physical time unit by multiplying the step size
$\Delta$. For a fBm self-similar traffic source,
Norros~\cite{norros95} provides its EB as
\begin{equation}\label{eq:norros}
  C = \lambda + (\kappa(H)\sqrt{-2\ln\epsilon})^{1/H}a^{1/(2H)}x^{-(1-H)/H}\lambda^{1/(2H)}
\end{equation}
where $\kappa(H) = H^H(1-H)^{(1-H)}$. Simplicity in the calculation is
the attractive feature of (\ref{eq:norros}).

The MVA technique developed in~\cite{kim01} so far provides the most
accurate estimation of the loss probability compared to previous
bandwidth allocation techniques according to simulation results.
Consider a discrete-time queueing system with constant service rate
$C$ and input process $\lambda_n$ with $\mathbb{E}\{\lambda_n\} =
\lambda$ and $\mathrm{Var}\{\lambda_n\} = \sigma^2$.  Define $X_n \equiv
\sum_{k=1}^n \lambda_k - Cn$.  The loss probability due to the MVA
approach is given by
\begin{equation}\label{eq:loss_mva}
  \varepsilon \approx \alpha e^{-m_x/2}
\end{equation}
where
\begin{equation}\label{eq:mx}
m_x = \min_{n \geq 0} \frac{((C-\lambda)n + B)^2}{\mathrm{Var}\{X_n\}} =
\frac{((C-\lambda)n^\ast + B)^2}{\mathrm{Var}\{X_{n^{\ast}}\}}
\end{equation} 
and 
\begin{equation}\label{eq:alpha}
  \alpha =
  \frac{1}{\lambda\sqrt{2\pi\sigma^2}}\exp\left(\frac{(C-\lambda)^2}{2\sigma^2}\right)
  \int_C^\infty (r-C)\exp\left(\frac{(r-\lambda)^2}{2\sigma^2}\right)\, dr
\end{equation}
For a given $\varepsilon$, we numerically solve for $C$ that satisfies
(\ref{eq:loss_mva}). Any search algorithm can be used to do the task.
Here, the bisection method is used.  

Next, we show how $\mathrm{Var}\{X_n\}$ can be determined.  Let
$C_{\lambda}(l)$ be the autocovariance function of $\lambda_n$.  The
MVA technique basically approximates the input process $\lambda_n$
with a Gaussian process, which allows $\mathrm{Var}\{X_n\}$ to be
represented by the autocovariance function.  In particular, the
variance of $X_n$ can be expressed in terms of $C_{\lambda}(l)$ as
\begin{equation}
  \mathrm{Var}\{X_n\} = nC_{\lambda}(0) + 2\sum_{l=1}^{n-1} (n-l)C_{\lambda}(l)
\end{equation} 
Therefore, $C_{\lambda}(l)$ must be known in the MVA technique, either
by assuming specific traffic models or by off-line analysis in case of
traces.  In most practical situations, $C_{\lambda}(l)$ will not be
known in advance, and an on-line measurement algorithm developed
in~\cite{eun01} is required to jointly determine both $n^\ast$ and
$m_x$. For fGn traffic, $\mathrm{Var}\{X_n\}$ is equal to $\sigma^2
n^{2H}$, where $\sigma^2 = \mathrm{Var}\{\lambda_n\}$, and we can find
the $n^\ast$ that minimizes (\ref{eq:mx}) directly. Although $\lambda$
can be easily measured, it is not the case for $\sigma^2$ and $H$.
Consequently, the MVA technique suffers from the need of prior
knowledge traffic parameters.


%%%%%%%%%%%%%%%%%%%%%%%%%%%%%%%%%%%%%%%%%%%%%%%%%%%%%%%%%%
%%%%%%%%%%%%%%% The 2nd appendix %%%%%%%%%%%%%%%%%%%%%%%%%%
%%%%%%%%%%%%%%%%%%%%%%%%%%%%%%%%%%%%%%%%%%%%%%%%%%%%%%%%%%
\appendix{ชื่อภาคผนวกที่ 2}
\noindent{\large\bf ใส่หัวข้อตามความเหมาะสม} \\


 \begin{figure}[!h]
\caption{This is the figure x11 ทดสอบ จาก \href{https://www.google.com} {https://www.google.com}}\label{fig:x1}
\end{figure}

Next, we show how $\mathrm{Var}\{X_n\}$ can be determined.  Let
$C_{\lambda}(l)$ be the autocovariance function of $\lambda_n$.  The
MVA technique basically approximates the input process $\lambda_n$
with a Gaussian process, which allows $\mathrm{Var}\{X_n\}$ to be
represented by the autocovariance function.  In particular, the
variance of $X_n$ can be expressed in terms of $C_{\lambda}(l)$ as
\begin{equation}
  \mathrm{Var}\{X_n\} = nC_{\lambda}(0) + 2\sum_{l=1}^{n-1} (n-l)C_{\lambda}(l)
\end{equation} 

\noindent{\large\bf Add more topic as you need} \\

Therefore, $C_{\lambda}(l)$ must be known in the MVA technique, either
by assuming specific traffic models or by off-line analysis in case of
traces.  In most practical situations, $C_{\lambda}(l)$ will not be
known in advance, and an on-line measurement algorithm developed
in~\cite{eun01} is required to jointly determine both $n^\ast$ and
$m_x$. For fGn traffic, $\mathrm{Var}\{X_n\}$ is equal to $\sigma^2
n^{2H}$, where $\sigma^2 = \mathrm{Var}\{\lambda_n\}$, and we can find
the $n^\ast$ that minimizes (\ref{eq:mx}) directly. Although $\lambda$
can be easily measured, it is not the case for $\sigma^2$ and $H$.
Consequently, the MVA technique suffers from the need of prior
knowledge traffic parameters. 





\end{document}
